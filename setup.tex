%======================文献加载格式
\usepackage[backend=biber,style=gb7714-2015, gbalign=gb7714-2015,gbnamefmt =lowercase]{biblatex}
\addbibresource{refs.bib}
%=========信息录入
\xnxyInfo{
 session={2024届},
 id ={20201405xxxx}, 
 displaytitle={毕业论文(设计)},
 title={基于\LaTeX 的两月半研究 \\ ---以xxx为例}, 
 author={xxx},
 department={数学与信息科学学院},
 major={计算机专业}, 
 supervisor={指导老师}, 
 assocsupervisor={副导师}, 
 date={2024年5月}
}

\usepackage{enumitem} %item设置
\usepackage[hidelinks]{hyperref}
\hypersetup{
    colorlinks=true, % 启用颜色链接
    linkcolor=black,   % 内部链接颜色
    citecolor=green, % 引用链接颜色
    filecolor=magenta, % 文件链接颜色
    urlcolor=red        % 外部 URL 链接颜色
}
\usepackage{graphicx}
\graphicspath{{figures/}}
\DeclareGraphicsExtensions{.pdf,.eps,.png,.jpg,.jpeg}
\usepackage{ntheorem}

\let \slantint \int
\usepackage{cmupint}%直立体
\usepackage{pifont}%雪花星符号
\usepackage{booktabs} %三线表

\usepackage{subcaption}
\usepackage{amssymb}
\usepackage{xcolor}
\NewDocumentCommand{\emphasize}{m}{{\color{blue}\textbf{#1}}}
\usepackage[linesnumbered,ruled,vlined]{algorithm2e}
\usepackage{listings}
\RenewDocumentCommand{\lstlistingname}{}{代码}
\lstdefinestyle{Matlab}{
language={Matlab},
basicstyle=\zihao{5}\ttfamily,
columns=flexible,
frame=single, framerule=1pt,
backgroundcolor=\color{white},
keywordstyle=\color{blue},
commentstyle=\color{green!50!black}, % 移除 \itshape
stringstyle=\color{blue},
breaklines=true, % 允许自动换行
showspaces=false, % 不显示空格
showstringspaces=false,% string不显示空格
numbers=left, % 在左侧显示行号
numberstyle=\zihao{5}, % 行号的字体大小
numbersep=5pt, % 行号与代码的距离
xleftmargin=3em, xrightmargin=3em,
captionpos=b,% 题注在下
}
\lstdefinestyle{LaTex}{
language=[LaTeX]TeX,
basicstyle=\zihao{5}\sffamily,
columns=flexible,
frame=single, framerule=1pt,
backgroundcolor=\color{white},
keywordstyle=\color{blue},
commentstyle=\color{green!75!black}, % 移除 \itshape
stringstyle=\color{blue},
breaklines=true, % 允许自动换行
showspaces=false, % 不显示空格
showstringspaces=false,% string不显示空格
xleftmargin=3em, xrightmargin=3em,
captionpos=b,% 题注在下
}


%=========绘图包
\usepackage{tikz}
\usetikzlibrary{arrows.meta}
\usepackage[siunitx]{circuitikz}%电路图
\usepackage{fancybox}
\usetikzlibrary{calc}

%======某些地方需要伪粗体
\usepackage{xfakebold}
\usepackage{xeCJKfntef}%下划线兼容问题

\setcounter{tocdepth}{2} % 设置目录深度为2