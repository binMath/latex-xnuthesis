\section{数学}
\LaTeX 公式符号系统比较完整,不会基本可查,作者虽是数学系但所选数学论文中公式使用的也较少,但还是总结了一些值得注意的坑。
\subsection{数学符号}
\subsubsection{分数}
使用行内公式会显得小,而使用\verb|\dfrac| 会感觉太挤了, 由于行间距一般不能改变,所以要么调行间距,要么使用行间公式,看下面示例
\begin{center}
    \begin{minipage}[c]{4em}
    有志登山顶$\dfrac{3}{9}$, 无志站$\frac{0}{9}$。
\end{minipage}
\end{center}
一般情况分数上下太宽建议直接放在行间。
\subsubsection{积分符号}
一些情况下我们可能需要直立体,而不是斜体。

 牛顿-莱布尼茨公式:
\[ \slantint_a^b f(x)\, dx=F(b)-F(a) , \forall x \in [a,b],F^{\prime}(x)=f(x) \]


高斯定理(散度定理)可以表示为:
\[
\oiint\limits_{\partial V} \mathbf{F} \cdot d\mathbf{S} = \iiint\limits_{V} \nabla \cdot \mathbf{F} \, dV
\]

其中,
\begin{itemize}
    \item $\oiint\limits_{\partial V} \mathbf{F} \cdot d\mathbf{S}$ 表示向量场 $\mathbf{F}$ 通过闭合表面 $\partial V$ 的向外通量,
    \item $\iiint\limits_{V} \nabla \cdot \mathbf{F} \, dV$ 表示向量场 $\mathbf{F}$ 在体积 $V$ 内的散度的积分。
\end{itemize}


\subsubsection{对齐点}
这里align与aligned完全是两个环境,aligned不是一个公式环境,

align:
\begin{align}
 a & = b + c \\
 & = d + e 
 \end{align}
 
 aligned:
 \begin{equation}
 \begin{aligned}
 a&= b+ c\\
 d&= e+ f+g\\
 h+i&= j+k\\
 l+m&= n
 \end{aligned}
 \end{equation}
 对齐需要对齐点,一般在\& 处对齐。
 \subsection{证明、定理和公理}
 本模板序号都用section编号,如果想使用单个序号1,2,3等,当然也可以通过类文件自定义。
 \begin{corollary}
 生活可能不像你想象的那么好,但是也不会像你想象的那么糟,人的脆弱和坚强都超乎了自己的想象,有时候脆弱的一句话会让你泪流满面,有时候你发现自己咬着牙已经走过了很长的路。
 \end{corollary}
 \begin{theorem}[三角形的内积和]
  两直角的平方差一定小于正弦的30°的一半。
 \end{theorem}


\begin{exercise}
子曰:打架用砖乎!不亦乱乎!\hfill ---《论语》
\end{exercise}
\begin{lemma}
 可导函数的每一个可导的极值点都是驻点(函数的导数在该点为零)。
\end{lemma}

\begin{proof}
     利用\[i +j =m\overrightarrow{a}\],可以得到\[\lim_{x+y} =C++\]
\end{proof}
 
\begin{example}[新高考\MakeUppercase{\romannumeral 2}]
想象有$n$个有序排列的箱子,其中每个箱子可以放一个球或者不放球。令二项式系数(或称为组合数) 
$m$ 个球的情况种数。

\end{example}

\begin{solution}
    利用组合数,
    \[ 
      C^m_n= \binom{n}{m} = \frac{n!}{m!(n-m)!}
    \]
    我们可以归纳得出...
\end{solution}


 \clearpage