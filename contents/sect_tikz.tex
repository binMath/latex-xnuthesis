\newpage
\section{TikZ}
与TikZ有关的\LaTeX 中宏包非常丰富,常规的流程图(附录C图\ref{tikzpicture:process}),函数图,电路图等都有对应的宏包支持,
它的语法也不同于Tex。入门TikZ建议看视频,这一块不建议上来就磕文档,官方文档有上千页。当然其绘图功能也是十分强大!

\begin{figure}[h]
    \centering
    \input{tikz/merge-sort-recursion-tree}
    \caption{二叉树}
    \label{tikzpic:tree}
\end{figure}
\begin{figure}[h]
    \centering
     \input{tikz/pie-chart-color}
    \caption{扇形图}
    \label{tikzpic:pie}
\end{figure}
\newpage
\begin{figure}[t]
     \centering
     \usetikzlibrary{arrows,shapes,calc,positioning}
\begin{circuitikz}[american]
\draw (0,0) node [scale=1.5,transformer core
] (T){}
      (T.A1) node[above] {A1}
      (T.A2) node[below] {A2}
      (T.B1) node[above left] {B1}
      (T.B2) node[below left] {B2}
      (T.base) node{};
\path (T.B1) -| ++ (2,1) coordinate (tb1){};% define a coordinate at top right relative to B1
\draw (T.B1) -- ++ (0,1) to[D*,-*]  (tb1);
\draw (T.B2) to[D*] ++(2,0)coordinate(tba){} -| (tba |- tb1);
\path (T.B2) |- ++(3,-1) coordinate(tb2){}; % define a coordinate at bottom right relative to B2
\draw (tb2) to[D*,*-] ++(-3,0) -- (T.B2);
\draw (tb2) --(tb2 |- T.B1) to[D*] (T.B1);
% place voltage labels
\draw(T.A1) to[open,v<={$240V_{rms}$,o-o}](T.A2);
\draw($(T.B1)!0.3!(T.B2)$) node[]{$12V_{rms,AC}$}(T.B1);
\draw ($(T.B1)!0.8!(T.B2)$)node[]{$12V_{rms,AC}$}(T.B2);
\draw[thick] ($(T.B1)!0.51!(T.B2)-(1cm,0)$)coordinate[](c){} -- ++ (12.44,0) -| ++ (1,-1) node[ground]{};                          % add a gorund to neutral line
% place capacitors and resistor (upper branches)
\draw(6,1) node[](d1){} to [C,l_=$C_1$, *-*] (c -| d1);
\draw(10,1)node[](d2){} to [C,l_=$C_3$, *-*] (c -| d2);
\draw(13,1)node[](d3){} to [R,l=$R_L$]       (c -| d3);
% place rectangles
\draw (7.5,0.5)  rectangle (8.5,1.5) node[below left= 0.25cm and 0.1cm]{780s};
\draw (7.5,-3.6) rectangle (8.5,-4.6)node[above left=0.25cm and 0.1cm]{790s};
% place capacitors and resistors (lower branches)
\draw(6,-4.15) node[](d4){} to [C,l_=$C_2$, *-*] (c -| d4);
\draw(10,-4.15)node[](d5){} to [C,l_=$C_4$, *-*] (c -| d5);
\draw(13,-4.15)node[](d6){} to [R,l=$R_L$]       (c -| d6);
\draw (8,0.5)  node[](d7){} to [short,-*]        (c -| d7) --(8,-3.6);
% place top and bottom lines
\draw(2,1) -- (7.5,1) (8.5,1) --(11,1) to[short,i^={$I_L$}] (13,1);
\draw(2,-4.15) -- (7.5,-4.15) (8.5,-4.15) --(13,-4.15);
\end{circuitikz}

    \caption{电路图}
    \label{tikzpic:circuit}
\end{figure}

上面的图形都是TikZ代码绘制的,TikZ功能远不止此,由于国内视频缺乏
完整教程,所以学习起来会有些吃力,作者也只是触碰到它的冰山一角,
图\ref{tikzpic:tree} 和图\ref{tikzpic:pie}来自
\url{https://texample.net/tikz},图\ref{tikzpic:circuit}转
载自\href{https://www.latexstudio.net/index/details/index/mid/2302}{latexstudio}。