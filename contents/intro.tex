\section{引言}
\subsection{为什学~\LaTeXe}
虽然word处理文字方便,但想要获得一个准确的样式却很难,特别是在公式与图片较多的时候会存在很多问题,而\LaTeX 排版数学公式非常高效,你一定没见过
\begin{equation}
\int_{a}^{b} \frac{d}{dx}\left( \frac{\sqrt{x^4 - 3x^2 + 2x + 1}}{\ln(x) - \sin(x)} \right) dx =
\left. \sum_{n=1}^{\infty} \frac{1}{n^2 + nx} \right|_{a}^{b} +
\prod_{i=1}^{n} i^2 - \bigg( \frac{\cos(x) + \sin(x)}{e^{x^2} - 1} \bigg)^{\frac{3}{2}} 
\tag{\ding{100}}
\end{equation}
这行公式仅需3行代码就搞定了。现在就带你一起来看看吧!
\subsection{\LaTeX 如何安装}
MiKTex 和Texlive都是主流软件,网上都有教程这里就不细说了。编辑体验比较好的有\href{https://cn.overleaf.com}{OverLeaf}和
VSCode。OverLeaf浏览器打开即用,本模板就由该网站提供的工具所作,VSCode需要配置环境,它的主要在输入提示上非常友好,不过这一点似乎TexLive支持的也挺好,VSCode适合有编程基础的同学。


\section{\LaTeX 快速入门}
作者也不是什么LaTeX专家,只能说是入门比较快的新手,浅浅分享一波经验!
新手当然是推荐看视频了,如果你是码农,那么直接上文档!
\begin{enumerate}
    \item 《一份不太简短的\LaTeXe 介绍》这本书做的不错,翻阅了无数遍,甚至写这篇文章还在用,入门前觉得像词典,入门后感觉是比较简洁的,放下它的github仓库地址:~\url{https://github.com/CTeX-org/lshort-zh-cn}。
    \item AI: 国内的通义千问,国外有许多免费ChatGPT-4站点可用,个人一直在用的\url{https://www.coze.com}挺不错的,作为新手几乎有一半的问题是靠它解决的。唯一不足之出就是大模型写代码还是一些存在bug的。
    \item 各类开源网站论坛
    \begin{itemize}
        \item \href{https://www.latexstudio.net/}{LaTex工作室}:该工作室在b站上的视频也值得去看。
        \item  \href{https://tex.stackexchange.com}{Tex.StackexChange}:一个专注解决Tex问题的 ``stackoverflow''。
        \item \href{https://ctan.org/}{CTAN} :开源宏包海洋,入海,做一个\LaTeX 极客。
    \end{itemize}
    \item  高校毕业论文模板,如\url{https://github.com/sjtug/SJTUThesis},是上海交通大学的论文模板,这个项目用了\href{https://www.latex-project.org/latex3/}{\LaTeX3}语法,是一个非常新的项目。入门的话更推荐北航,天津大学等的模板,它们的类文件语法偏Tex。
\end{enumerate}


