{\centering
\section{说明书}}

C++是一种被广泛使用的计算机程序设计语言。它是一种通用程序设计语言,支持多重编程范式,例如过程化程序设计、面向对象程序设计、泛型程序设计和函数式程序设计等。

比雅尼·斯特劳斯特鲁普博士在贝尔实验室工作期间在20世纪80年代发明并实现了C++。起初,这种语言被称作“C with Classes”(“包含‘类’的C语言”),作为C语言的增强版出现。随后,C++不断增加新特性。虚函数、运算符重载、多继承、标准模板库、异常处理、运行时类型信息、命名空间等概念逐渐纳入标准草案。1998年,国际标准组织颁布了C++程序设计语言的第一个国际标准ISO/IEC 14882:1998,目前最新标准为ISO/IEC 14882:2020。ISO/IEC 14882通称ISO C++。ISO C++包含了主要包含了核心语言和标准库的规则。尽管从核心语言到标准库都有显著不同,ISO C++直接正式(normative)引用了ISO/IEC 9899(通称ISO C),且ISO C++标准库的一部分和ISO C的标准库的API完全相同,另有很小一部分和C标准库略有差异(例如,strcat等函数提供对const类型的重载)。这使得C和C++的标准库实现常常被一并提供,在核心语言规则很大一部分兼容的情况下,进一步确保用户通常较容易把符合ISO C的源程序不经修改或经极少修改直接作为C++源程序使用,也是C++语言继C语言之后流行的一个重要原因。

作为广泛被使用的工业语言,C++存在多个流行的成熟实现:GCC、基于LLVM的Clang以及Visual C++等。这些实现同时也是成熟的C语言实现,但对C语言的支持程度不一(例如,VC++对ANSI C89之后的标准支持较不完善)。大多数流行的实现包含了编译器和C++部分标准库的实现。编译器直接提供核心语言规则的实现,而库提供ISO C++标准库的实现。这些实现中,库可能同时包含和ISO C标准库的共享实现(如VC++的msvcrt);而另一些实现的ISO C标准库则是单独于编译器项目之外提供的,如glibc和musl。C++标准库的实现也可能支持多种编译器,如GCC的libstdc++库支持GCC的g++和LLVM Clang的clang++。这些不同的丰富组合使市面上的C++环境具有许多细节上的实现差异,因而遵循ISO C++这样的权威标准对维持可移植性显得更加重要。现今讨论的C++语言,除非另行指明,通常均指ISO C++规则定义的C++语言(虽然因为实现的差异,可能不一定是最新的正式版本)。




\newpage