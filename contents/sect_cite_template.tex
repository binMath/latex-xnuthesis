\section{文献引用与\LaTeX 模板}
\subsection{文献}
本文参考文献的格式为GB/T7714-2015,使用bibtex进行管理,\verb|\cite{}|对文献引用:
\begin{center}

单文件引用\cite{Cheng1999},多个文献引用\cite{CSTAM1990,GBT2659,HBLZ2001,Hopkinson1999,Jiang1989,Jiang1998,Li2000,Li1999,LSC1957,WHO1970,Yu2001,Zhang1998}。
    
\end{center}
\subsection{模板}
xnuthesis论文结构大致如下,更多使用方法可以参考本文的源码,如需学习
撰写类文件可以看\url{https://github.com/rockyzhz/latexdoc-chinese-translation}提供的文档 ,有许多翻译后的中文文档可以参考。

\begin{figure}[p]
\centering
\lstinputlisting[style=LaTex,caption={LaTex论文结构}]{main.tex}
\end{figure}

\LaTeXe{}对类文件开发维护困难,主要体现在许多命名规范上的问题 ,于是就产生了\LaTeX 3,\LaTeX 3语法就是由expl3宏包提供的,\url{https://texdoc.org/serve/interface3/0}是\LaTeX 3 语法的完整手册。\LaTeX 3更像是一门面向开发者的编程语言,有类型并且支持函数式编程,命名规范,
缺点是目前还未完全普及,许多宏包还没支持新的l3语法,但这是未来的发展趋势。为了以后兼容性,本模板在类文件中部分也使用了\LaTeX3语法。

\section{总结}
本模板的许多结构方面参考了北航和上交的模板,作者全程依赖GPT解决了许多疑问,不得不说的是
人工智能时代,信息获取方式越来越快,AI已经能实现模块化的基本需求,学习和写出 \LaTeX 程序变得更加轻松了,Typst 与 MathML发展日益强大,也希望某一天能出现更优秀简洁排版方式吧!

\newpage
