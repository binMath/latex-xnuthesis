%-------------------自己写的封面非官方
\newcommand{\ulineshort}[1]{\underline{\makebox[9em][c]{#1}}}
\newcommand{\ulinelong}[1]{\underline{\makebox[22em][c]{\zihao{3} \kaishu #1}}}

%华文中宋与宋体相差不大,所以用宋体加伪粗体代替了
\newcommand{\zhongsongbf}[1]{\zihao{-3}  \songti \setBold[0.4] #1 \unsetBold} 
\newcommand{\displaytitlefont}[1]{\zihao{-0} \songti  \setBold[0.4] #1 \unsetBold }

%=========封面
\begin{titlepage}
\begin{tikzpicture}[remember picture, overlay]
\coordinate (SessionIDinfo) at (\textwidth,0);
%---------每个部分以页面中心为相对位置
\coordinate (Topic) at  (current page.center);
\coordinate (BasicInfo) at ($(current page.center)!0.5!(current page.south)$);
\coordinate (Dissertname) at ($(current page.center)!0.25!(current page.north)$);
\coordinate (xnxybadge) at ($(current page.center)!0.5!(current page.north)$);
\coordinate (xnxyName) at ($(xnxybadge)!0.5!(Dissertname)$);


\node[anchor=north east] at (SessionIDinfo) {
\begin{tabular}{l}
    \zihao{5} \fangsong 届 \quad 别\ulineshort{\makebox[9em][c]{\tlsession}} \\
    \zihao{5} \fangsong 学 \quad 号\ulineshort{\makebox[9em][c]{\tlid} } \\
\end{tabular}
};
%==============论文主要元素
\node at (xnxybadge) {
    \includegraphics[scale=0.08]{cover/xnxyLogo}
    };
\node at (xnxyName) {
    \includegraphics[scale=0.2]{cover/xnxyName}
    };
\node at (Dissertname) { \displaytitlefont{ \tldisplaytitle }};
%============1/2处论文标题
\node at (Topic) {
\begin{minipage}[c]{0.8\textwidth}
   \centering{ \zihao{-2} \CJKunderline{\tltitle}}
\end{minipage}

};

%===========基本信息
\node at (BasicInfo) {
\begin{tabular}{c@{}l}
    \zhongsongbf{姓\qquad \quad \qquad 名 }  & \ulinelong{\tlauthor }\\[0.5cm]
    \zhongsongbf{学\quad 院、专 \quad 业 }    & \ulinelong{\tldepartment } \\[0.5cm]
                                             & \ulinelong{\tlmajor }  \\[0.5cm]
     \zhongsongbf{导师姓名、职称}          &\ulinelong{\tlsupervisor \quad \tlassocsupervisor}\\[0.5cm]
     \zhongsongbf{完\quad 成\quad 时\quad 间 }& \ulinelong{\tldate } \\[0.5cm]
\end{tabular}
};
\end{tikzpicture}
\end{titlepage}

    
